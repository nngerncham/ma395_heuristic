\section{Tabu Search with Frequency}

\subsection{Implementation}

Due to some unforeseen circumstances, my implementation of TS for the previous problems does not support what I want to do for this problem anymore. So, I reimplemented TS but with very minor difference in logic from the one used for previous problems. The full implementation can be found \href{https://github.com/nngerncham/ma395_heuristic/blob/main/homework/hw3/codebase/tabu_search/p4.py}{here} and the Jupyter Notebook used can be found \href{https://github.com/nngerncham/ma395_heuristic/blob/main/homework/hw3/codebase/Notebooks/Problem\%204.ipynb}{here}.

The policy for handling the aspiration criterion is that every 30 iterations, it will replace the cost function with the score or \(\epsilon\) function to sort \(V^*\). Then, the value used to compare with the aspiration level is also computed by \(\epsilon\).

\subsection{Experiments}

For each variant of tabu list and set of parameters in this problem, each experiment is done 100 times. However, the number of iterations run is different for each algorithm so please check the plot carefully. Additionally, the terminologies may be a bit weird so I will explain it here:
\begin{itemize}
    \item Swap list refers to the tabu list variant that stores and checks both elements/modules of a swap
    \item Store(ing) min refers to the tabu list variant that stores the element/module with lower index among the two elements in a single swap operation
\end{itemize}

In Figures \ref{fig:p4-min}, \ref{fig:p4-freq}, and \ref{fig:p4-tweaked-freq}, we show the progresses of each variant of TS in comparison to the swap list with frequency. For storing min and swap list without frequency, the parameters used were \(|T| = 4\) and \(|V^*| = 3\). Same goes with the untweaked swap list with frequency. The tweaked swap list with frequency uses \(|V^*| = 6\), however. The evaluation function \(\epsilon\) is the same as in the handout. Namely,
\[
    \epsilon(H, s^*) = \begin{cases}
        \text{cost}(s^*) & \text{cost}(s^*) \leq \text{cost}(s) \\
        \text{cost}(s^*) + 5 \times \text{Freq}(i, j) & \text{otherwise}
    \end{cases}
\]

Note that the figure only shows the progress plot for storing min since storing the min or max does not have that big of a difference in terms of the progress.

\begin{figure}[ht]
    \centering
    \hfill
    \begin{subfigure}{0.48\textwidth}
        \centering
        \includegraphics[width=\textwidth]{../images/p4/store-min.png}
        \caption{Storing min}
    \end{subfigure}
    \hfill
    \begin{subfigure}{0.48\textwidth}
        \centering
        \includegraphics[width=\textwidth]{../images/p4/store-min-pi.png}
        \caption{Storing min (range)}
    \end{subfigure}
    \hfill
    \caption{Storing only storing one element of a swap}
    \label{fig:p4-min}
\end{figure}

\begin{figure}[ht]
    \hfill
    \begin{subfigure}{0.48\textwidth}
        \centering
        \includegraphics[width=\textwidth]{../images/p4/store-pair.png}
        \caption{Storing swap}
    \end{subfigure}
    \hfill
    \begin{subfigure}{0.48\textwidth}
        \centering
        \includegraphics[width=\textwidth]{../images/p4/store-freq-base.png}
        \caption{Storing swap with frequency}
    \end{subfigure}
    \hfill
    \caption{Storing swap with and without frequency}
    \label{fig:p4-freq}
\end{figure}

\begin{figure}[ht]
    \hfill
    \begin{subfigure}{0.48\textwidth}
        \centering
        \includegraphics[width=\textwidth]{../images/p4/store-freq-tweaked.png}
        \caption{Storing swap with frequency, tweaked}
    \end{subfigure}
    \hfill
    \begin{subfigure}{0.48\textwidth}
        \centering
        \includegraphics[width=\textwidth]{../images/p4/store-freq-tweaked-pi.png}
        \caption{Storing swap with frequency, tweaked (range)}
    \end{subfigure}
    \hfill

    \caption{Tweaked swap store with frequency}
    \label{fig:p4-tweaked-freq}
\end{figure}

We can see in Figure \ref{fig:p4-min} that while storing the min element always approaches the exact optimum within 1000 iterations, it does not guarantee it. In contrast, we can see that the tweaked swap list with frequency always converges within less than 300 iterations in Figure \ref{fig:p4-tweaked-freq}. Notice also that the standard deviation for storing swap with frequency and storing min are about the same but the mean is lower.

We will hypothesize that the decrease in objective function cost comes from the fact that \(\epsilon\) enforces more exploration just by making sure that infrequently made moves will have lower scores. This means that TS is more likely to go through more elements in the feasible region even if the actual cost is higher than optimum, leading to faster convergence and more accurate results.
