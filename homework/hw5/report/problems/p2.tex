\section{Shaffer's F2 Problem}

\subsection{Finding the Frontiers}

Let us first compute the objective values of each point:
\begin{table}[ht]
    \centering
    \begin{tabular}{cccccc}
        \toprule
        Point & A & B & C & D & E \\
        \midrule
        \(f_1, f_2\) & (0.22, 2.35) & (3.01, 0.07) & (0.67, 1.40) & (0.17, 5.83) & (10.31, 1.47) \\
        \toprule
        Point & F & G & H & I & J \\
        \midrule
        \(f_1, f_2\) & (1.62, 10.71) & (2.27, 12.31) & (3.36, 14.68) & (4.67, 17.31) & (16.85, 37.27) \\
        \bottomrule
    \end{tabular}
\end{table}

Then, we can construct a similar table as the last problem as follows:

\begin{table}[ht]
    \centering
    \caption{Table of points, the points it dominates, and number of points it's dominated by}
    \begin{tabular}{ccc}
        \toprule
        \textbf{Point} & \textbf{Dominates} & \textbf{Dominated Count} \\
        \midrule
        A & F, G, H, I, J & 0 \\
        B & E, H, I, J & 0 \\
        C & E, F, G, H, I, J & 0 \\
        D & F, G, H, I, J & 0\\
        E & J & 2 \\
        F & G, H, I, J & 3 \\
        G & H, I, J & 4 \\
        H & I, J & 5 \\
        I & I, & 6 \\
        J & - & 7 \\
        \bottomrule
    \end{tabular}
\end{table}

Thus, we have the following frontiers:
\[
\begin{aligned}
    \mathcal{F}_1 &= \left\{A, B, C, D\right\} \\
    \mathcal{F}_2 &= \left\{E, F\right\} \\
    \mathcal{F}_3 &= \left\{G, H\right\} \\
    \mathcal{F}_4 &= \left\{I, J\right\}
\end{aligned}
\]

\newpage

\subsection{Plot of the Fronts}

\begin{figure}[ht]
    \centering
    \includegraphics[width=0.8\textwidth]{images/p2-pareto_front.png}
    \caption{The frontiers of Shaffer's F2 Problems}
\end{figure}

\subsection{Calculating Crowd Distances}

Since every front except for \(\mathcal{F}_1\) has only two points, we can ignore them as the crowding distances of all points would be \(\infty\). Thus, we will only focus on \(\mathcal{F}_1\).

First, let us sort points of \(\mathcal{F}_1\) based on their objective functions.
\[
\begin{aligned}
    f_1 &: [D, A, C, B] \\
    f_2 &: [B, C, A, D]
\end{aligned}
\]
Notice that
\[
\begin{aligned}
    f_1^{\max} &= 16.85 &\quad& f_1^{\min} &= 0.17 \\
    f_2^{\max} &= 37.27 &\quad& f_2^{\min} &= 0.07
\end{aligned}
\]
Now, we can compute the crowding distances for each of the points.
\[
\begin{aligned}
    cd(A) &= \left(\frac{f_1(C) - f_1(D)}{f_1^{\max} - f_1^{\min}}\right) + \left(\frac{f_2(D) - f_2(C)}{f_2^{\max} - f_2^{\min}}\right) \\
          &= \left(\frac{0.67 - 0.17}{16.85 - 0.17}\right) + \left(\frac{5.83 - 1.4}{37.27 - 0.07}\right) \\
          &\approx 0.149 \\
    cd(B) &= \infty \\
    cd(C) &= \left(\frac{f_1(B) - f_1(A)}{f_1^{\max} - f_1^{\min}}\right) + \left(\frac{f_2(A) - f_2(B)}{f_2^{\max} - f_2^{\min}}\right) \\
          &= \left(\frac{3.01 - 0.22}{16.85 - 0.17}\right) + \left(\frac{2.35 - 0.07}{37.27 - 0.07}\right) \\
          &\approx 0.229 \\
    cd(D) &= \infty
\end{aligned}
\]

\subsection{Using Tournament Selection}

\begin{enumerate}[label=(\alph*.)]
    \item \textbf{C vs. G} Between C and G, C would win since it has lower rank. That is, it is a lower frontier.
    \item \textbf{A vs. C} Between A and C, C would win because while they have the same rank, C has higher crowding distance at 0.229 compared to A's 0.149.
\end{enumerate}
