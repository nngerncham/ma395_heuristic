\documentclass{article}

% BASE TAKEN FROM ICCS315 SCRIBE NOTES

% --- SETUP STUFF ---
\usepackage[a4paper, margin=1in]{geometry}
\usepackage{booktabs}
\usepackage{url}

% --- MATH STUFF ---
\usepackage{amsthm, amsmath, amssymb}
\usepackage{mathtools,xspace}
\usepackage{nicefrac}

\usepackage{bbm}
\usepackage{dsfont}
\usepackage{cancel}

\usepackage{blkarray}
\newcommand{\matindex}[1]{\mbox{\scriptsize#1}}% Matrix index

% --- FONT STUFF ---
% Has to be under math stuff for some reason :/
\usepackage{newpxtext, newpxmath}
\usepackage[T1]{fontenc}

% --- DIAGRAM STUFF ---
\usepackage{tikz,pgfplots}
\usepackage{xcolor}
\usepackage{graphicx}
\usepackage{colortbl}
\usepackage{caption}
\usepackage{subcaption}

\usepackage{hyperref}

\usepackage{tcolorbox}
\usepackage{framed}

\usepackage{algorithm}
\usepackage[indLines=true]{algpseudocodex}

\pgfplotsset{compat=1.18}

% --- THEOREM STUFF ---
\newtheorem{theorem}{Theorem}[section]
\newtheorem{proposition}[theorem]{Proposition}
\newtheorem{lemma}[theorem]{Lemma}
\newtheorem{corollary}[theorem]{Corollary}

\theoremstyle{definition}
\newtheorem{definition}[theorem]{Definition}

\theoremstyle{remark}
\newtheorem{remark}[theorem]{Remark}
\newtheorem{claim}[theorem]{Claim}
\newtheorem{fact}[theorem]{Fact}

\input{macros.tex}

\title{\Huge{Homework 1}
	\\
	\Large\scshape{Heuristic Optimization}}
\author{Nawat Ngerncham}
\date{\today}

\begin{document}

\maketitle
	
\section{Implementation Details}

The full implementation of each algorithm can be found \hyperlink{https://github.com/nngerncham/ma395_heuristic/tree/main/homework/hw1/tsp_algorithms}{here}. Following are short descriptions of what each algorithm is.

\begin{itemize}
    \item \textbf{Random Sampling} ---
        In each iteration, RS shuffles the cities in the path around and checks if the distance goes down or not

    \item \textbf{Exhaustive 2-swap} ---
        Starting from a random initial path, TS generates every combination of cities that can be swapped, performs the swap, and checks against current best found solution if the distance goes down or not
    \item \textbf{Randomized 2-swap} ---
        Similarly to TS, RTS starts from a random initial path. However, it randomly selects ONE pair of cities to swap, performs the swap, then compares the distance and updates the best found solution

    \item \textbf{Exhaustive 2-opt} --- 
        Starting from a random initial path, TO generates all valid (non-adjacent) pairs of cities in the path then tests if the path distance goes down or not in each iteration after the transformation
    \item \textbf{Randomized 2-opt} --- 
        Similarly to TO, RTO starts from a random initial path. The difference is that in each iteration, RTO randomly picks ONE valid pair of cities in the path and performs the tranformation then compares the distances
\end{itemize}

\section{Experiment Details}

The results presented here are from 30 trials. Each algorithm runs 1500, 2500, and 10000 iterations in each trial for GR17, FRI26, and ATT48, respectively.

\section{Results}

\begin{figure}[h]
    \begin{subfigure}[b]{0.3\textwidth}
        \includegraphics[width=\textwidth]{images/gr17_plot.png}
        \caption{GR17}
    \end{subfigure}
    \hfill
    \begin{subfigure}[b]{0.3\textwidth}
        \includegraphics[width=\textwidth]{images/fri26_plot.png}
        \caption{FRI26}
    \end{subfigure}
    \hfill
    \begin{subfigure}[b]{0.3\textwidth}
        \includegraphics[width=\textwidth]{images/att48_plot.png}
        \caption{ATT48}
    \end{subfigure}
    \caption{Progress line on each data set}
\end{figure}


\section{Discussion}

From the plots above, we can see that exhaustive or randomized, all of these algorithms will almost always get stuck in a local minimum at some point. (The true optimal solution is the black line at the bottom of each plot.) Random sampling is the only exception since it never reached anywhere near the true optimum and its rate of convergence is straight up horrible. So, we will mainly focus on 2-opt and 2-swap for the rest of this discussion.

Now, onto the results of exhaustive versions of 2-swap and 2-opt. On every data set, it is clear that 2-opt always performs far better than 2-swap. I suspect that this is because the transformation in each iteration of 2-opt changes two edges instead of 2-swap's four. As a result, it is able to make smaller changes and explore the search space a bit more---increasing the likelihood that it hits a better approximate optimum. However, it is still affected by the local minimum problem. Still, the local minimum that 2-opt reaches is still better than any other algorithm, exhaustive or randomized.

This observation is untrue for randomized versions, however. Randomized 2-swap and 2-opt each starts off performing roughly about the same, i.e. reaching similar local minimums in GR17. However, the gap between their performance increases as the size of the data set becomes larger to the point that it becomes very significant in ATT48.

\end{document}
