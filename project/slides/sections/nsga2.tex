\section{MOO-ing with NSGA-II(-ish)}

\frame{\insertsection}

\subsection{Setup}

\begin{frame}{Encoding the Chromosomes for NSGA-II}
    For NSGA-II, one individual consists of 4 strings representing each decision variable. They have the following transformation:
    \begin{itemize}
        \item For \(M\), its string has length 10 and uses normal positive integer-to-binary conversion
        \item For \(C\) and \(S\), their strings also has length 10 and uses normal positive integer-to-binary conversion with the condition that if the conversion falls below 100, it defaults to 100
        \item For \(\alpha\), since it is a real-valued parameter, we convert the string to an integer, say \(x\), then do the following:
            \[
                x \mapsto 1 + \frac{x}{1025}
            \]
            Note that the denominator is 1025 not 1024 because the interval \(\alpha\) belongs to is \([1, 2)\)
    \end{itemize}
\end{frame}

\begin{frame}{Crossover and Selection}
    The combination of crossover and selection methods used were:
    \begin{itemize}
        \item Single cut-catenate crossover with uniform random selection
        \item Multiple cut-catenate crossover with uniform random selection
        \item Single cut-catenate crossover with tournament selection
        \item Multiple cut-catenate crossover with tournament selection
    \end{itemize}
\end{frame}

\subsection{Normalization}

\begin{frame}{Nadir and Ideal Point}
    Let \(X^*\) denote the pareto front and suppose that we have \(m\) objective functions. We have the following:
    \begin{itemize}
        \item Nadir point \(z_i^{nad}\) is formally defined by
            \[
                z^{nad} = \begin{pmatrix}
                    \sup_{x^* \in X^*} f_1(x^*) \\
                    \vdots \\
                    \sup_{x^* \in X^*} f_m(x^*) \\
                \end{pmatrix}
            \]
        \item Ideal point \(z_i^*\) is formally defined by
            \[
                z^{*} = \begin{pmatrix}
                    \inf_{x^* \in X^*} f_1(x^*) \\
                    \vdots \\
                    \inf_{x^* \in X^*} f_m(x^*) \\
                \end{pmatrix}
            \]
    \end{itemize}
    In this project, the best-found-so-far are used to estimate both points.
\end{frame}

\begin{frame}{Normalization}
    The normalization method that we will use is simply as follows: For the \(i\)-th objective value denoted \(f_i\), we normalize it to \(\bar{f_i}\) by
    \[
        \bar{f_i} = \frac{f_i - \hat{z}^*_i}{\hat{z}_i^{nad} - \hat{z}_i^*}
    \]

    Then, we have that \(\bar{f_i} \in [0, 1]\) for any \(i\).
\end{frame}

\subsection{Results}

\begin{frame}{Scaling vs. Non-scaling}
    \begin{figure}
        \centering
        \hfill
        \begin{subfigure}{0.49\textwidth}
            \includegraphics[width=\textwidth]{../images/report/non-scaling.png}
            \caption{Non-scaling}
        \end{subfigure}
        \hfill
        \begin{subfigure}{0.49\textwidth}
            \includegraphics[width=\textwidth]{../images/report/scaling.png}
            \caption{Scaling}
        \end{subfigure}
    \end{figure}
\end{frame}
