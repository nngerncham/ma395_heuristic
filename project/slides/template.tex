\documentclass{beamer}

% BASE TAKEN FROM ICCS315 SCRIBE NOTES

% --- SETUP STUFF ---
\usepackage[a4paper, margin=1in]{geometry}
\usepackage{booktabs}
\usepackage{url}

% --- MATH STUFF ---
\usepackage{amsthm, amsmath, amssymb}
\usepackage{mathtools,xspace}
\usepackage{nicefrac}

\usepackage{bbm}
\usepackage{dsfont}
\usepackage{cancel}

\usepackage{blkarray}
\newcommand{\matindex}[1]{\mbox{\scriptsize#1}}% Matrix index

% --- FONT STUFF ---
% Has to be under math stuff for some reason :/
\usepackage{newpxtext, newpxmath}
\usepackage[T1]{fontenc}

% --- DIAGRAM STUFF ---
\usepackage{tikz,pgfplots}
\usepackage{xcolor}
\usepackage{graphicx}
\usepackage{colortbl}
\usepackage{caption}
\usepackage{subcaption}

\usepackage{hyperref}

\usepackage{tcolorbox}
\usepackage{framed}

\usepackage{algorithm}
\usepackage[indLines=true]{algpseudocodex}

\pgfplotsset{compat=1.18}

% --- THEOREM STUFF ---
\newtheorem{theorem}{Theorem}[section]
\newtheorem{proposition}[theorem]{Proposition}
\newtheorem{lemma}[theorem]{Lemma}
\newtheorem{corollary}[theorem]{Corollary}

\theoremstyle{definition}
\newtheorem{definition}[theorem]{Definition}

\theoremstyle{remark}
\newtheorem{remark}[theorem]{Remark}
\newtheorem{claim}[theorem]{Claim}
\newtheorem{fact}[theorem]{Fact}

\input{macros.tex}

\title{MOO-ing at Expensive Real-world Functions}
\author{Nawat Ngerncham}
\date{\today}

\begin{document}

\begin{frame}
  \titlepage
\end{frame}

\begin{frame}{Overview}
  \tableofcontents
\end{frame}

\section{Background}

\begin{frame}{\(k\)-Nearest Neighbor Search}
\begin{definition}[\(k\)-Nearest Neighbor Search]
    Given the point set \(\mathcal{P} \subseteq \mathbb{R}^d\), query point \(q \in \mathbb{R}^d\), and a distance function \(\delta\), we want to find a set \(\mathcal{K} \subseteq \mathcal{P}\) such that \(|\mathcal{K}| = k\) and
    \[
        \max_{p \in \mathcal{K}} \delta(p, q) \leq \min_{p \in \mathcal{P} \setminus \mathcal{K}} \delta(p, q)
    \]
    \end{definition}
    \begin{itemize}
        \item Real-world problems where \(k\)-NNS is applied include information retrieval systems and recommendation systems
        \item Finding exact \(\mathcal{K}\) is expensive, so we usually approximate it instead
    \end{itemize}
\end{frame}

\begin{frame}{Existing \textit{Solutions}}
    There are a few approaches to solving (Approximate) \(k\)-NNS:
    \begin{itemize}
        \item Naive linear search (too long to compute)
        \item Hash-based like LSH (does not work in high \(d\))
        \item Tree-based like \(k\)D-trees (also does not work in high \(d\))
        \item Graph-based (where the good stuff is)
    \end{itemize}
\end{frame}

\begin{frame}{Graph-based ANNS Algorithms}
    Out of all the approaches mentioned, graph-based approaches have shown the best results. Usually, graph-based solutions have 2 main phases:
    \begin{enumerate}
        \item Index construction (building the graph)
        \item Search the index (search the graph)
    \end{enumerate}
\end{frame}

\begin{frame}{DiskANN and Vamana}
    \begin{itemize}
        \item DiskANN is a system that allows us to perform ANNS tasks
        \item Vamana is the algorithm used to build to the graph used in DiskANN
        \item I might (informally) use these interchangeably
    \end{itemize}
\end{frame}

\begin{frame}
\begin{algorithm}[H]
    \caption{GreedySearch Algorithm}\label{alg:greedy-search}
    \begin{algorithmic}[1]
        \Function{GreedySearch}{\(G, s, q, k, S\){}}
            \State{\(\mathcal{L} \gets \left\{s\right\}\); \(\mathcal{V} \gets \emptyset\)}
            \While{\(\mathcal{L} \setminus \mathcal{V} \neq \emptyset\)}
                \State{\(p^* \gets \arg\min_{p \in \mathcal{L} \setminus \mathcal{V}} \delta(p, q)\)}
                \State{\(\mathcal{L} \gets \mathcal{L} \cup N_{\text{out}}(p^*)\)}
                \State{\(\mathcal{V} \gets \mathcal{V} \cup \left\{p^*\right\}\)}
                \If{\(|\mathcal{L}| > S\)}
                    \State{\(\mathcal{L} \gets \textsc{GetClosest}(\mathcal{L}, S)\)}
                \EndIf
            \EndWhile
            \State{\Return{\(\mathcal{L}, \mathcal{V}\)}}
        \EndFunction
    \end{algorithmic}
\end{algorithm}

\textbf{Inputs:} Graph \(G\), Starting point \(s\), Query point \(q\), Number of neighbors \(k\), Candidate set size \(S\)
\end{frame}

\begin{frame}
\begin{algorithm}[H]
    \caption{RobustPrune Algorithm}\label{alg:robust-prune}
    \begin{algorithmic}[1]
        \Procedure{RobustPrune}{\(G, p, \mathcal{V}, \alpha, M\){}}
            \State{\(\mathcal{V} \gets \mathcal{V} \cup N_{\text{out}}(p) \setminus \left\{p\right\};N_\text{out}(p) \gets \emptyset\)}
            \While{\(\mathcal{V} \neq \emptyset\)}
                \State{\(p^* \gets \arg\min_{p \in \mathcal{V}} \delta(p, q)\)}
                \State{\(N_\text{out}(p) \gets N_\text{out}(p) \cup \left\{p^*\right\}\)}
                \If{\(|N_\text{out}(p) = M\)}
                    \State{break}
                \EndIf
                \For{\(p' \in \mathcal{V}\)}
                    \If{\(\alpha \cdot \delta(p^*, p') \leq \delta(p, p')\)}
                        \State{\(\mathcal{V} \gets \mathcal{V} \setminus \left\{p'\right\}\)}
                    \EndIf
                \EndFor
            \EndWhile
        \EndProcedure
    \end{algorithmic}
\end{algorithm}

\textbf{Inputs:} Graph \(G\), Starting point \(p\), Candidate set \(\mathcal{V}\), Distance scaler \(\alpha\), Max degree \(M\)
\end{frame}

\begin{frame}
\vspace{-4mm}
\begin{algorithm}[H]
    \caption{Vamana Algorithm}\label{alg:vamana}
    \begin{algorithmic}[1]
        \Function{VamanaBuild}{\(\mathcal{P}, \alpha, C, M\){}}
            \State{Initialize \(G\) as random directed graph with max degree \(M\)}
            \State{\(s \gets \textsc{Medoid}(\mathcal{P})\); \(n \gets |\mathcal{P}|\)}
            \For{\(p \in \mathcal{P}\)}
                \State{\(\mathcal{L}, \mathcal{V} \gets \textsc{GreedySearch}(s, p, 1, C)\)}
                \State{\(\textsc{RobustPrune}(p, \mathcal{V}, \alpha, M)\)}
                \For{\(p' \in N_\text{out}(p)\)}
                    \If{\(|N_\text{out}(p') \cup \left\{p'\right\}| > M\)}
                        \State{\(\textsc{RobustPrune}(p', N_\text{out}(p')\cup\left\{p'\right\}, \alpha, M)\)}
                    \Else
                        \State{\(N_\text{out}(p') \gets N_\text{out}(p') \cup \left\{p'\right\}\)}
                    \EndIf
                \EndFor
            \EndFor
            \State{\Return{\(G\)}}
        \EndFunction
    \end{algorithmic}
\end{algorithm}

\textbf{Inputs:} Point set \(\mathcal{P}\), Dist. scaler \(\alpha\), Candidate size \(C\), Max degree \(M\)
\end{frame}

\begin{frame}{Parameters of DiskANN}
    From the three algorithms, we can identify the following parameters:
    \begin{itemize}
        \item \(S\) from \textsc{GreedySearch}
        \item \(C\) from \textsc{VamanaBuild}
        \item \(M\) from \textsc{VamanaBuild}
        \item \(\alpha\) from \textsc{RobustPrune} and \textsc{VamanaBuild}
    \end{itemize}

    These parameters affect the construction time, search time, and search quality (recall) the graph. An obvious question arises: How should we set them? Let's use Optimization to answer that question!
\end{frame}

\section{Formulation}

\begin{frame}{Decision Variables}
    Consider the following decision variables:
    \[
    \begin{aligned}
        M &= \text{Maximum degree of \textit{each} vertex in \(G\)} \\
        C &= \text{Size of the candidate set during construction} \\
        S &= \text{Size of the candidate set during search} \\
        \alpha &= \text{Distance threshold for \textsc{RobustPrune}}
    \end{aligned}
    \]
    
    We then apply the following constraints:
    \[
    \begin{aligned}
        &1 \leq M \leq 1024, & M \in \mathbb{Z} \\
        &100 \leq C \leq 1024, & C \in \mathbb{Z} \\
        &100 \leq S \leq 1024, & S \in \mathbb{Z} \\
        &1 \leq \alpha < 2, &\alpha \in \mathbb{R}
    \end{aligned}
    \]
\end{frame}

\begin{frame}{Objective Functions}
    Let \(bp\) denote a configuration of the parameters. Consider the following objective functions:
    \[
    \begin{aligned}
        f_c(bp) &= \text{Time taken for constructing the graph, to minimize} \\
        f_s(bp) &= \text{Time taken to search the queries in total, to minimize} \\
        f_r(bp) &= \text{Quality of the search results, to maximize}
    \end{aligned}
    \]

    We will treat these objective functions as black boxes.
\end{frame}

\begin{frame}{Full model}
    \[
    \begin{aligned}
        \min_{bp} &\left(f_c(bp), f_s(bp), 1 - f_r(bp)\right) \\
        \text{s.t.} \qquad
            &1 \leq M \leq 1024, & M \in \mathbb{Z} \\
            &100 \leq C \leq 1024, & C \in \mathbb{Z} \\
            &100 \leq S \leq 1024, & S \in \mathbb{Z} \\
            &1 \leq \alpha < 2, &\alpha \in \mathbb{R}
    \end{aligned}
    \]
\end{frame}

\section{MOO-ing with NSGA-II}

\subsection{Setup}

\begin{frame}{Encoding the Chromosomes for NSGA-II}
    For NSGA-II, one individual consists of 4 strings representing each decision variable. They have the following transformation:
    \begin{itemize}
        \item For \(M\), its string has length 10 and uses normal positive integer-to-binary conversion
        \item For \(C\) and \(S\), their strings also has length 10 and uses normal positive integer-to-binary conversion with the condition that if the conversion falls below 100, it defaults to 100
        \item For \(\alpha\), since it is a real-valued parameter, we convert the string to an integer, say \(x\), then do the following:
            \[
                x \mapsto 1 + \frac{x}{1025}
            \]
            Note that the denominator is 1025 not 1024 because the interval \(\alpha\) belongs to is \([1, 2)\)
    \end{itemize}
\end{frame}

\subsection{Normalization}

\begin{frame}{Nadir and Ideal Point}
    Let \(X^*\) denote the pareto front and suppose that we have \(m\) objective functions. We have the following:
    \begin{itemize}
        \item Nadir point \(z_i^{nad}\) is formally defined by
            \[
                z^{nad} = \begin{pmatrix}
                    \sup_{x^* \in X^*} f_1(x^*) \\
                    \vdots \\
                    \sup_{x^* \in X^*} f_m(x^*) \\
                \end{pmatrix}
            \]
        \item Ideal point \(z_i^*\) is formally defined by
            \[
                z^{*} = \begin{pmatrix}
                    \inf_{x^* \in X^*} f_1(x^*) \\
                    \vdots \\
                    \inf_{x^* \in X^*} f_m(x^*) \\
                \end{pmatrix}
            \]
    \end{itemize}
    In this project, the best-found-so-far are used to estimate both points.
\end{frame}

\begin{frame}{Normalization}
    The normalization method that we will use is simply as follows: For the \(i\)-th objective value denoted \(f_i\), we normalize it to \(\bar{f_i}\) by
    \[
        \bar{f_i} = \frac{f_i - \hat{z}^*_i}{\hat{z}_i^{nad} - \hat{z}_i^*}
    \]

    Then, we have that \(\bar{f_i} \in [0, 1]\) for any \(i\).
\end{frame}

\section{Conclusion}

\end{document}
